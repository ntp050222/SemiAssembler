\chapter{Introduction}

    High-throughput sequencing technologies, especially the next-generation sequencing (NGS), have been widely deployed in recent years. The genomes of more and more species have been sequenced and decoded, such as mouse ~\cite{Waterston2002}, human ~\cite{Venter2001}, and rice~\cite{IRGSP}. In spite of the powerful sequencing technologies, it remains a computationally challenge to reconstruct a complete genome, mainly owing to complex repeat structures in the genome and relative short length of reads~\cite{Dunham}. The genomes assembled by well-known assemblers (e.g., Velvet, ABySS and SOAPdenovo) are often fragmented into large numbers of contigs~\cite{Zerbino2008,Simpson2009,Li2008}. 

    Instead of {\em de novo} assembling a genome, some tools generate consensus sequences based on a closely-related reference genome (e.g., SAMtools vcf2fq)~\cite{Samtoolsa,Samtoolsb}. However, the differences between any two genomes increase with respect to their evolutionary distance. In fact, large-scale structural variants (SVs), including single nucleotide polymorphisms (SNPs), insertions, deletions, and inversions~\cite{Mills2006}, are often found between the genomes of two closely-related species (e.g., human and chimpanzee). Although the generated consensus sequences contain SNPs, other large variations will not be identified via this reference-mapping approach. 

   In recent years, many methods have been developed to identify SVs within populations of the same species via reference-mapping approach (e.g., NovelSeq, SOAPindel, MindTheGap)~\cite{Hajirasouliha2010,Shengting2013,Guillaume2014}. However, these methods have their own strength and weakness for detecting small-sized or large-sized SVs. NovelSeq is limited to novel insertions, and SOAPindel is limited to short insertions. And all of them mainly report the SV locus or SV sequences but fail to reconstruct the entire genome. 

   As many genomes have been assembled, the newly-sequenced genomes are often closely-related to an existing genome. For instance, divergence between the human and chimpanzee genomes is only 5\%~\cite{Roy2002}. Therefore, the {\em de novo} assembly of new species (with genomes of close-related species available) is not a cost-effective approach. {\em De novo} assembly has the advantage of assembling SVs, and reference-mapping approaches maintain the genome integrity. As a consequence, hybrid approaches able to integrate the strengths of {\em de novo} assembly and reference-mapping approaches are still highly demanded.

% modified
In this thesis, we design a semi-assembly approach called SemiAssembler which integrate reference-mapping approaches and {\em de novo} assembly to reconstruct a newly-sequenced genome using closely-related reference genome. Given closely-related reference sequences and NGS pair reads, we detect possible locus of insertions and deletions by capturing the signatures (e.g., aberrant mapping distance, breakpoint reads) of insertions/deletions from aligning NGS pair reads onto reference genome. Moreover, we can estimate the size of large-sized insertions and the range of large-sized deletions. We assemble paired-end reads to contigs by de novo assembler, and then try to assemble large-sized insertions from these contigs. A draft genome is created by adding novel insertion sequences and by removing deleted sequences to and from the draft genome, respectively. Subsequently, the draft genome sequence is replaced with the contig sequences, which is able to reflect inter-species SNPs and small-sized indels. Finally, we can get a newly-sequenced genome which provides better contiguity but also uncover a substantial amount of inter-species variations. The experimental results were carried out in both simulated data sets and biological findings.

    The rest of this thesis is organized as follows. Chapter 2 provides a review of related literature used by this study. In Chapter 3, we illustrate the material and methods used in our study. In Chapter 4, we demonstrate the experimental results of our study. Final,chapter 5 summarizes conclusions and future works.